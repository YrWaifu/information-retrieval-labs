\section{Цель и общая постановка задач}

Целью серии лабораторных работ по курсу «Информационный поиск» является изучение
основных этапов построения простейшей поисковой системы. В рамках работ
необходимо сформировать корпус документов, подготовить его для последующей
обработки, проанализировать статистические свойства текстов, реализовать
индексирование и поиск по документам, а также оценить эффективность работы
получившейся системы.

Выполняемые лабораторные работы логически связаны друг с другом:
\begin{itemize}
    \item в первой работе подготавливается корпус документов, описываются его
    характеристики и источники данных;
    \item во второй работе анализируются статистические свойства текста
    (в том числе закон Ципфа);
    \item в третьей и четвёртой работах реализуется индексация корпуса и
    булев поиск по нему;
    \item в пятой работе проводится экспериментальная оценка эффективности
    построенной поисковой системы.
\end{itemize}

\pagebreak
